% Options for packages loaded elsewhere
\PassOptionsToPackage{unicode}{hyperref}
\PassOptionsToPackage{hyphens}{url}
\documentclass[
]{article}
\usepackage{xcolor}
\usepackage[margin=1in]{geometry}
\usepackage{amsmath,amssymb}
\setcounter{secnumdepth}{-\maxdimen} % remove section numbering
\usepackage{iftex}
\ifPDFTeX
  \usepackage[T1]{fontenc}
  \usepackage[utf8]{inputenc}
  \usepackage{textcomp} % provide euro and other symbols
\else % if luatex or xetex
  \usepackage{unicode-math} % this also loads fontspec
  \defaultfontfeatures{Scale=MatchLowercase}
  \defaultfontfeatures[\rmfamily]{Ligatures=TeX,Scale=1}
\fi
\usepackage{lmodern}
\ifPDFTeX\else
  % xetex/luatex font selection
\fi
% Use upquote if available, for straight quotes in verbatim environments
\IfFileExists{upquote.sty}{\usepackage{upquote}}{}
\IfFileExists{microtype.sty}{% use microtype if available
  \usepackage[]{microtype}
  \UseMicrotypeSet[protrusion]{basicmath} % disable protrusion for tt fonts
}{}
\makeatletter
\@ifundefined{KOMAClassName}{% if non-KOMA class
  \IfFileExists{parskip.sty}{%
    \usepackage{parskip}
  }{% else
    \setlength{\parindent}{0pt}
    \setlength{\parskip}{6pt plus 2pt minus 1pt}}
}{% if KOMA class
  \KOMAoptions{parskip=half}}
\makeatother
\usepackage{color}
\usepackage{fancyvrb}
\newcommand{\VerbBar}{|}
\newcommand{\VERB}{\Verb[commandchars=\\\{\}]}
\DefineVerbatimEnvironment{Highlighting}{Verbatim}{commandchars=\\\{\}}
% Add ',fontsize=\small' for more characters per line
\usepackage{framed}
\definecolor{shadecolor}{RGB}{248,248,248}
\newenvironment{Shaded}{\begin{snugshade}}{\end{snugshade}}
\newcommand{\AlertTok}[1]{\textcolor[rgb]{0.94,0.16,0.16}{#1}}
\newcommand{\AnnotationTok}[1]{\textcolor[rgb]{0.56,0.35,0.01}{\textbf{\textit{#1}}}}
\newcommand{\AttributeTok}[1]{\textcolor[rgb]{0.13,0.29,0.53}{#1}}
\newcommand{\BaseNTok}[1]{\textcolor[rgb]{0.00,0.00,0.81}{#1}}
\newcommand{\BuiltInTok}[1]{#1}
\newcommand{\CharTok}[1]{\textcolor[rgb]{0.31,0.60,0.02}{#1}}
\newcommand{\CommentTok}[1]{\textcolor[rgb]{0.56,0.35,0.01}{\textit{#1}}}
\newcommand{\CommentVarTok}[1]{\textcolor[rgb]{0.56,0.35,0.01}{\textbf{\textit{#1}}}}
\newcommand{\ConstantTok}[1]{\textcolor[rgb]{0.56,0.35,0.01}{#1}}
\newcommand{\ControlFlowTok}[1]{\textcolor[rgb]{0.13,0.29,0.53}{\textbf{#1}}}
\newcommand{\DataTypeTok}[1]{\textcolor[rgb]{0.13,0.29,0.53}{#1}}
\newcommand{\DecValTok}[1]{\textcolor[rgb]{0.00,0.00,0.81}{#1}}
\newcommand{\DocumentationTok}[1]{\textcolor[rgb]{0.56,0.35,0.01}{\textbf{\textit{#1}}}}
\newcommand{\ErrorTok}[1]{\textcolor[rgb]{0.64,0.00,0.00}{\textbf{#1}}}
\newcommand{\ExtensionTok}[1]{#1}
\newcommand{\FloatTok}[1]{\textcolor[rgb]{0.00,0.00,0.81}{#1}}
\newcommand{\FunctionTok}[1]{\textcolor[rgb]{0.13,0.29,0.53}{\textbf{#1}}}
\newcommand{\ImportTok}[1]{#1}
\newcommand{\InformationTok}[1]{\textcolor[rgb]{0.56,0.35,0.01}{\textbf{\textit{#1}}}}
\newcommand{\KeywordTok}[1]{\textcolor[rgb]{0.13,0.29,0.53}{\textbf{#1}}}
\newcommand{\NormalTok}[1]{#1}
\newcommand{\OperatorTok}[1]{\textcolor[rgb]{0.81,0.36,0.00}{\textbf{#1}}}
\newcommand{\OtherTok}[1]{\textcolor[rgb]{0.56,0.35,0.01}{#1}}
\newcommand{\PreprocessorTok}[1]{\textcolor[rgb]{0.56,0.35,0.01}{\textit{#1}}}
\newcommand{\RegionMarkerTok}[1]{#1}
\newcommand{\SpecialCharTok}[1]{\textcolor[rgb]{0.81,0.36,0.00}{\textbf{#1}}}
\newcommand{\SpecialStringTok}[1]{\textcolor[rgb]{0.31,0.60,0.02}{#1}}
\newcommand{\StringTok}[1]{\textcolor[rgb]{0.31,0.60,0.02}{#1}}
\newcommand{\VariableTok}[1]{\textcolor[rgb]{0.00,0.00,0.00}{#1}}
\newcommand{\VerbatimStringTok}[1]{\textcolor[rgb]{0.31,0.60,0.02}{#1}}
\newcommand{\WarningTok}[1]{\textcolor[rgb]{0.56,0.35,0.01}{\textbf{\textit{#1}}}}
\usepackage{graphicx}
\makeatletter
\newsavebox\pandoc@box
\newcommand*\pandocbounded[1]{% scales image to fit in text height/width
  \sbox\pandoc@box{#1}%
  \Gscale@div\@tempa{\textheight}{\dimexpr\ht\pandoc@box+\dp\pandoc@box\relax}%
  \Gscale@div\@tempb{\linewidth}{\wd\pandoc@box}%
  \ifdim\@tempb\p@<\@tempa\p@\let\@tempa\@tempb\fi% select the smaller of both
  \ifdim\@tempa\p@<\p@\scalebox{\@tempa}{\usebox\pandoc@box}%
  \else\usebox{\pandoc@box}%
  \fi%
}
% Set default figure placement to htbp
\def\fps@figure{htbp}
\makeatother
\setlength{\emergencystretch}{3em} % prevent overfull lines
\providecommand{\tightlist}{%
  \setlength{\itemsep}{0pt}\setlength{\parskip}{0pt}}
\usepackage{bookmark}
\IfFileExists{xurl.sty}{\usepackage{xurl}}{} % add URL line breaks if available
\urlstyle{same}
\hypersetup{
  pdftitle={Turnover Impact and Win Probability Analysis: Rams vs Seahawks Week 11},
  pdfauthor={Navin Chandradat},
  hidelinks,
  pdfcreator={LaTeX via pandoc}}

\title{Turnover Impact and Win Probability Analysis: Rams vs Seahawks
Week 11}
\author{Navin Chandradat}
\date{December 16, 2025}

\begin{document}
\maketitle

{
\setcounter{tocdepth}{2}
\tableofcontents
}
\section{Introduction}\label{introduction}

When two really good teams meet you expect a really good game. When they
play in the same division sometimes things can go awry. But what we got
here was two really great teams on both sides of the ball and the
matchup did not disappoint. Now you could say that on offence a lot went
wrong for the Seahawks and while they ultimately lost the game, they
only lost by 2 points and could have won if a really long field goal
goes through the uprights. Now I wanted to explore two questions: 1) Did
the Rams capitalize on the turnovers on defence and 2) Because there
were so many turnovers, how did the win probability flow over the course
of this game? As mentioned before, these are divisional opponents and
they've only played one time so hopefully this analysis ends up being
useful for when they meet again.

\begin{center}\rule{0.5\linewidth}{0.5pt}\end{center}

\begin{Shaded}
\begin{Highlighting}[]
\CommentTok{\# Loading in our required packages}
\CommentTok{\# nflfastr}
\FunctionTok{library}\NormalTok{(nflreadr)}
\FunctionTok{library}\NormalTok{(tidyverse)}
\FunctionTok{library}\NormalTok{(nflplotR)}
\end{Highlighting}
\end{Shaded}

\begin{Shaded}
\begin{Highlighting}[]
\CommentTok{\# Load 2025 season data}
\NormalTok{pbp\_2025 }\OtherTok{\textless{}{-}} \FunctionTok{load\_pbp}\NormalTok{(}\DecValTok{2025}\NormalTok{)}
\FunctionTok{head}\NormalTok{(pbp\_2025)}
\end{Highlighting}
\end{Shaded}

\begin{verbatim}
## # A tibble: 6 x 372
##   play_id game_id      old_game_id home_team away_team season_type  week posteam
##     <dbl> <chr>        <chr>       <chr>     <chr>     <chr>       <int> <chr>  
## 1       1 2025_01_ARI~ 2025090705  NO        ARI       REG             1 <NA>   
## 2      40 2025_01_ARI~ 2025090705  NO        ARI       REG             1 ARI    
## 3      63 2025_01_ARI~ 2025090705  NO        ARI       REG             1 ARI    
## 4      85 2025_01_ARI~ 2025090705  NO        ARI       REG             1 ARI    
## 5     115 2025_01_ARI~ 2025090705  NO        ARI       REG             1 ARI    
## 6     135 2025_01_ARI~ 2025090705  NO        ARI       REG             1 ARI    
## # i 364 more variables: posteam_type <chr>, defteam <chr>, side_of_field <chr>,
## #   yardline_100 <dbl>, game_date <chr>, quarter_seconds_remaining <dbl>,
## #   half_seconds_remaining <dbl>, game_seconds_remaining <dbl>,
## #   game_half <chr>, quarter_end <dbl>, drive <dbl>, sp <dbl>, qtr <dbl>,
## #   down <dbl>, goal_to_go <int>, time <chr>, yrdln <chr>, ydstogo <dbl>,
## #   ydsnet <dbl>, desc <chr>, play_type <chr>, yards_gained <dbl>,
## #   shotgun <dbl>, no_huddle <dbl>, qb_dropback <dbl>, qb_kneel <dbl>, ...
\end{verbatim}

\begin{Shaded}
\begin{Highlighting}[]
\CommentTok{\# Filter for our specific game}
\NormalTok{game\_data }\OtherTok{\textless{}{-}}\NormalTok{ pbp\_2025 }\SpecialCharTok{\%\textgreater{}\%}
  \FunctionTok{filter}\NormalTok{(}
\NormalTok{    week }\SpecialCharTok{==} \DecValTok{11}\NormalTok{,}
\NormalTok{    home\_team }\SpecialCharTok{==} \StringTok{"LA"}\NormalTok{,}
\NormalTok{    away\_team }\SpecialCharTok{==} \StringTok{"SEA"}
\NormalTok{  )}
\FunctionTok{head}\NormalTok{(game\_data)}
\end{Highlighting}
\end{Shaded}

\begin{verbatim}
## # A tibble: 6 x 372
##   play_id game_id      old_game_id home_team away_team season_type  week posteam
##     <dbl> <chr>        <chr>       <chr>     <chr>     <chr>       <int> <chr>  
## 1       1 2025_11_SEA~ 2025111609  LA        SEA       REG            11 <NA>   
## 2      39 2025_11_SEA~ 2025111609  LA        SEA       REG            11 LA     
## 3      62 2025_11_SEA~ 2025111609  LA        SEA       REG            11 LA     
## 4      89 2025_11_SEA~ 2025111609  LA        SEA       REG            11 LA     
## 5     125 2025_11_SEA~ 2025111609  LA        SEA       REG            11 LA     
## 6     147 2025_11_SEA~ 2025111609  LA        SEA       REG            11 LA     
## # i 364 more variables: posteam_type <chr>, defteam <chr>, side_of_field <chr>,
## #   yardline_100 <dbl>, game_date <chr>, quarter_seconds_remaining <dbl>,
## #   half_seconds_remaining <dbl>, game_seconds_remaining <dbl>,
## #   game_half <chr>, quarter_end <dbl>, drive <dbl>, sp <dbl>, qtr <dbl>,
## #   down <dbl>, goal_to_go <int>, time <chr>, yrdln <chr>, ydstogo <dbl>,
## #   ydsnet <dbl>, desc <chr>, play_type <chr>, yards_gained <dbl>,
## #   shotgun <dbl>, no_huddle <dbl>, qb_dropback <dbl>, qb_kneel <dbl>, ...
\end{verbatim}

\begin{Shaded}
\begin{Highlighting}[]
\CommentTok{\# Quick check}
\FunctionTok{cat}\NormalTok{(}\StringTok{"Total plays in game:"}\NormalTok{, }\FunctionTok{nrow}\NormalTok{(game\_data), }\StringTok{"}\SpecialCharTok{\textbackslash{}n}\StringTok{"}\NormalTok{)}
\end{Highlighting}
\end{Shaded}

\begin{verbatim}
## Total plays in game: 178
\end{verbatim}

\begin{Shaded}
\begin{Highlighting}[]
\FunctionTok{cat}\NormalTok{(}\StringTok{"Final Score {-} LA:"}\NormalTok{, }\FunctionTok{max}\NormalTok{(game\_data}\SpecialCharTok{$}\NormalTok{home\_score, }\AttributeTok{na.rm =} \ConstantTok{TRUE}\NormalTok{), }
    \StringTok{"SEA:"}\NormalTok{, }\FunctionTok{max}\NormalTok{(game\_data}\SpecialCharTok{$}\NormalTok{away\_score, }\AttributeTok{na.rm =} \ConstantTok{TRUE}\NormalTok{), }\StringTok{"}\SpecialCharTok{\textbackslash{}n}\StringTok{"}\NormalTok{)}
\end{Highlighting}
\end{Shaded}

\begin{verbatim}
## Final Score - LA: 21 SEA: 19
\end{verbatim}

\begin{center}\rule{0.5\linewidth}{0.5pt}\end{center}

\section{Part 1: Turnover Impact
Analysis}\label{part-1-turnover-impact-analysis}

This game was very intriguing because the Rams were able to take the
ball away five times, and yet were only able to win the game by 2
points. This raises the question: Did the Rams effectively capitalize on
their defensive turnovers? To answer this, I will analyze expected
points and drives following each turnover.

\subsection{Identifying Turnovers}\label{identifying-turnovers}

\begin{Shaded}
\begin{Highlighting}[]
\CommentTok{\# Identify all turnovers}
\CommentTok{\# game\_data \%\textgreater{}\%}
\CommentTok{\#      distinct(interception)}
\NormalTok{turnovers }\OtherTok{\textless{}{-}}\NormalTok{ game\_data }\SpecialCharTok{\%\textgreater{}\%}
  \FunctionTok{filter}\NormalTok{(interception }\SpecialCharTok{==} \DecValTok{1} \SpecialCharTok{|}\NormalTok{ fumble\_lost }\SpecialCharTok{==} \DecValTok{1}\NormalTok{) }\SpecialCharTok{\%\textgreater{}\%}
  \FunctionTok{select}\NormalTok{(}
\NormalTok{    qtr, time, down, ydstogo, yardline\_100, }
\NormalTok{    desc, posteam, defteam,}
\NormalTok{    interception, fumble\_lost,}
\NormalTok{    ep, epa, wp, wpa}
\NormalTok{  ) }\SpecialCharTok{\%\textgreater{}\%}
  \FunctionTok{arrange}\NormalTok{(qtr, }\FunctionTok{desc}\NormalTok{(time))}

\NormalTok{turnovers}
\end{Highlighting}
\end{Shaded}

\begin{verbatim}
## # A tibble: 5 x 14
##     qtr time   down ydstogo yardline_100 desc       posteam defteam interception
##   <dbl> <chr> <dbl>   <dbl>        <dbl> <chr>      <chr>   <chr>          <dbl>
## 1     1 10:59     1      10           79 (10:59) (~ SEA     LA                 1
## 2     3 14:19     3       9           64 (14:19) (~ SEA     LA                 1
## 3     3 14:09     2      10           35 (14:09) (~ LA      SEA                0
## 4     3 01:15     1      20           73 (1:15) 14~ SEA     LA                 1
## 5     4 10:55     3       3           36 (10:55) (~ SEA     LA                 1
## # i 5 more variables: fumble_lost <dbl>, ep <dbl>, epa <dbl>, wp <dbl>,
## #   wpa <dbl>
\end{verbatim}

\begin{Shaded}
\begin{Highlighting}[]
\FunctionTok{cat}\NormalTok{(}\StringTok{"Number of turnovers:"}\NormalTok{, }\FunctionTok{nrow}\NormalTok{(turnovers), }\StringTok{"}\SpecialCharTok{\textbackslash{}n}\StringTok{"}\NormalTok{)}
\end{Highlighting}
\end{Shaded}

\begin{verbatim}
## Number of turnovers: 5
\end{verbatim}

Taking a look at the turnovers, it seems like going into halftime the
Seahawks probably felt pretty good having committed only 1 turnover in
the first half and Rams committing none. The story changes going into
the second half of the game where the Seahawks committed 3 more, two of
them in the third quarter, on top of the Rams committing a turnover of
their own. Both teams in the second half committed a turnover that set
up the other team with great field position in opponent territory.

Now let's take a look at which turnovers had the biggest impact on a
teams probability of winning the game.

\begin{Shaded}
\begin{Highlighting}[]
\NormalTok{turnovers }\SpecialCharTok{\%\textgreater{}\%} 
     \FunctionTok{select}\NormalTok{(qtr, time, defteam, wpa) }\SpecialCharTok{\%\textgreater{}\%}
     \FunctionTok{arrange}\NormalTok{(}\FunctionTok{desc}\NormalTok{(}\FunctionTok{abs}\NormalTok{(wpa)))}
\end{Highlighting}
\end{Shaded}

\begin{verbatim}
## # A tibble: 5 x 4
##     qtr time  defteam     wpa
##   <dbl> <chr> <chr>     <dbl>
## 1     1 10:59 LA      -0.182 
## 2     3 01:15 LA      -0.173 
## 3     4 10:55 LA      -0.112 
## 4     3 14:09 SEA     -0.0919
## 5     3 14:19 LA      -0.0564
\end{verbatim}

Looking at win probability added (WPA values are negative because they
reflect the offense's perspective), the first interception that the
Seahawks threw had the largest impact to the team's chance at winning at
about 18.2\%. All turnovers are bad, but these numbers suggest that the
first interception set the tone for the rest of the game. This single
turnover had nearly double the impact of LA's only turnover at about
9.2\%.

\subsection{Expected Points Analysis}\label{expected-points-analysis}

\begin{Shaded}
\begin{Highlighting}[]
\CommentTok{\# Calculating expected points for each turnover}

\NormalTok{turnover\_analysis }\OtherTok{\textless{}{-}}\NormalTok{ turnovers }\SpecialCharTok{\%\textgreater{}\%}
  \FunctionTok{mutate}\NormalTok{(}
    \AttributeTok{turnover\_team =}\NormalTok{ defteam,}
    \AttributeTok{turnover\_location =}\NormalTok{ yardline\_100,}
    \AttributeTok{turnover\_number =} \FunctionTok{row\_number}\NormalTok{(),}
    \AttributeTok{expected\_points =}\NormalTok{ ep}
\NormalTok{  )}


\CommentTok{\#Change this to look at expect points PER TEAM}
\CommentTok{\# total\_expected \textless{}{-} sum(turnover\_analysis$expected\_points)}
\CommentTok{\# cat("Total Expected Points from turnovers:", round(total\_expected, 1), "\textbackslash{}n")}
\end{Highlighting}
\end{Shaded}

\begin{Shaded}
\begin{Highlighting}[]
\CommentTok{\# To get a summary of turnovers versus expected points}
\NormalTok{turnover\_analysis }\SpecialCharTok{\%\textgreater{}\%}
    \FunctionTok{group\_by}\NormalTok{(defteam) }\SpecialCharTok{\%\textgreater{}\%} \FunctionTok{summarise}\NormalTok{(}
        \AttributeTok{turnovers\_forced =} \FunctionTok{n}\NormalTok{(),}
        \AttributeTok{total\_expected\_points =} \FunctionTok{sum}\NormalTok{(expected\_points)}
\NormalTok{    )}
\end{Highlighting}
\end{Shaded}

\begin{verbatim}
## # A tibble: 2 x 3
##   defteam turnovers_forced total_expected_points
##   <chr>              <int>                 <dbl>
## 1 LA                     4                  5.58
## 2 SEA                    1                  3.61
\end{verbatim}

Taking a look at the numbers, the Rams forced 4 turnovers but only got
5.6 expected points, an average of 1.4 points per turnover. In theory,
this isn't great from the perspective of the Rams because you would
think they would've had more expected points given the number of
turnovers they had. If you dig a little bit deeper, you see that 3 of
the Rams' 4 turnovers were in their own territory. This poor field
position helps explain why, even though the Rams were able to force 4
turnovers, they only won the game by two points.

\subsection{Drive Analysis}\label{drive-analysis}

\begin{Shaded}
\begin{Highlighting}[]
\CommentTok{\# Analyze all drives from the game}
\NormalTok{drives }\OtherTok{\textless{}{-}}\NormalTok{ game\_data }\SpecialCharTok{\%\textgreater{}\%}
  \FunctionTok{filter}\NormalTok{(}\SpecialCharTok{!}\FunctionTok{is.na}\NormalTok{(posteam), posteam }\SpecialCharTok{!=} \StringTok{""}\NormalTok{) }\SpecialCharTok{\%\textgreater{}\%}  \CommentTok{\# Remove rows without offensive team}
  \FunctionTok{group\_by}\NormalTok{(fixed\_drive) }\SpecialCharTok{\%\textgreater{}\%}
  \FunctionTok{summarise}\NormalTok{(}
    \AttributeTok{qtr =} \FunctionTok{first}\NormalTok{(qtr),}
    \AttributeTok{posteam =} \FunctionTok{first}\NormalTok{(posteam),}
    \AttributeTok{drive\_start\_yard\_line =} \FunctionTok{first}\NormalTok{(yardline\_100),}
    \AttributeTok{plays =} \FunctionTok{n}\NormalTok{(),}
    \AttributeTok{yards =} \FunctionTok{sum}\NormalTok{(yards\_gained, }\AttributeTok{na.rm =} \ConstantTok{TRUE}\NormalTok{),}
    \AttributeTok{points\_scored =} \FunctionTok{last}\NormalTok{(posteam\_score\_post) }\SpecialCharTok{{-}} \FunctionTok{first}\NormalTok{(posteam\_score\_post),}
    \AttributeTok{drive\_result =} \FunctionTok{case\_when}\NormalTok{(}
      \FunctionTok{any}\NormalTok{(touchdown }\SpecialCharTok{==} \DecValTok{1}\NormalTok{) }\SpecialCharTok{\textasciitilde{}} \StringTok{"Touchdown"}\NormalTok{,}
      \FunctionTok{any}\NormalTok{(field\_goal\_attempt }\SpecialCharTok{==} \DecValTok{1} \SpecialCharTok{\&}\NormalTok{ field\_goal\_result }\SpecialCharTok{==} \StringTok{"made"}\NormalTok{) }\SpecialCharTok{\textasciitilde{}} \StringTok{"Field Goal"}\NormalTok{,}
      \FunctionTok{any}\NormalTok{(interception }\SpecialCharTok{==} \DecValTok{1} \SpecialCharTok{|}\NormalTok{ fumble\_lost }\SpecialCharTok{==} \DecValTok{1}\NormalTok{) }\SpecialCharTok{\textasciitilde{}} \StringTok{"Turnover"}\NormalTok{,}
      \FunctionTok{any}\NormalTok{(punt\_attempt }\SpecialCharTok{==} \DecValTok{1}\NormalTok{) }\SpecialCharTok{\textasciitilde{}} \StringTok{"Punt"}\NormalTok{,}
      \ConstantTok{TRUE} \SpecialCharTok{\textasciitilde{}} \StringTok{"Other"}
\NormalTok{    )}
\NormalTok{  ) }\SpecialCharTok{\%\textgreater{}\%}
  \FunctionTok{ungroup}\NormalTok{()}
\end{Highlighting}
\end{Shaded}

\begin{Shaded}
\begin{Highlighting}[]
\NormalTok{drives}
\end{Highlighting}
\end{Shaded}

\begin{verbatim}
## # A tibble: 24 x 8
##    fixed_drive   qtr posteam drive_start_yard_line plays yards points_scored
##          <dbl> <dbl> <chr>                   <dbl> <int> <dbl>         <dbl>
##  1           1     1 LA                         35     8    47             0
##  2           2     1 SEA                        92     3    13             0
##  3           3     1 LA                          3     5     3             7
##  4           4     1 SEA                        35     7    26             3
##  5           5     1 LA                         35    13    84             7
##  6           6     1 SEA                        35    17    43             3
##  7           7     2 LA                         35     5     1             0
##  8           8     2 SEA                        92    15    89             3
##  9           9     3 SEA                        35     4     1             0
## 10          10     3 LA                         35     2     7             0
## # i 14 more rows
## # i 1 more variable: drive_result <chr>
\end{verbatim}

\begin{Shaded}
\begin{Highlighting}[]
\CommentTok{\# For future reference because row 8 was an end of half situation there\textquotesingle{}s an error in the points scored entry }
\CommentTok{\# it should say 3 points scored}

\CommentTok{\# The drives that say other were either a turnover on downs, a kneeldown, or a field goal kick at the end of halves as time expired}
\end{Highlighting}
\end{Shaded}

Looking at the drives from the game, we can see that of the four drives
following takeaways the Rams were able to score 2 touchdowns resulting
in 14 points and the Seahawks were able to walk away with a field goal
(3 points) on the drive following their takeaway. Based on our
calculations from before we can see that the Rams were expected to score
approximately 5.6 points based on their field position after the
turnovers and the Seahawks were expected to score approximately 3.6
after their takeaway. This demonstrates that, despite poor field
position, the Rams were able to capitalize on their defensive turnovers
and converted them into two touchdowns on following drives. But it is
worth noting the extra possessions were pivotal in the Rams winning the
ball game because on all other offensive drives that didn't follow a
turnover, they were only able to score 7 points.

\subsection{Visualization: Turnover
Locations}\label{visualization-turnover-locations}

\begin{Shaded}
\begin{Highlighting}[]
\NormalTok{turnover\_analysis }\SpecialCharTok{\%\textgreater{}\%}
  \FunctionTok{ggplot}\NormalTok{(}\FunctionTok{aes}\NormalTok{(}\AttributeTok{x =}\NormalTok{ turnover\_number, }\AttributeTok{y =}\NormalTok{ turnover\_location)) }\SpecialCharTok{+}
  \FunctionTok{geom\_segment}\NormalTok{(}\FunctionTok{aes}\NormalTok{(}\AttributeTok{x =}\NormalTok{ turnover\_number, }\AttributeTok{xend =}\NormalTok{ turnover\_number,}
                   \AttributeTok{y =} \DecValTok{0}\NormalTok{, }\AttributeTok{yend =}\NormalTok{ turnover\_location,}
                   \AttributeTok{color =}\NormalTok{ turnover\_team),}
                \AttributeTok{size =} \DecValTok{2}\NormalTok{) }\SpecialCharTok{+}
  \FunctionTok{geom\_point}\NormalTok{(}\FunctionTok{aes}\NormalTok{(}\AttributeTok{size =}\NormalTok{ expected\_points, }\AttributeTok{color =}\NormalTok{ turnover\_team), }\AttributeTok{alpha =} \FloatTok{0.7}\NormalTok{)}\SpecialCharTok{+}
  \CommentTok{\# Set team colors}
  \FunctionTok{scale\_color\_manual}\NormalTok{(}\AttributeTok{values =} \FunctionTok{c}\NormalTok{(}\StringTok{"LA"} \OtherTok{=} \StringTok{"navy"}\NormalTok{, }\StringTok{"SEA"} \OtherTok{=} \StringTok{"green"}\NormalTok{))}\SpecialCharTok{+}
  \FunctionTok{scale\_y\_reverse}\NormalTok{(}\AttributeTok{breaks =} \FunctionTok{seq}\NormalTok{(}\DecValTok{0}\NormalTok{, }\DecValTok{100}\NormalTok{, }\DecValTok{10}\NormalTok{)) }\SpecialCharTok{+}
  \FunctionTok{labs}\NormalTok{(}
    \AttributeTok{title =} \StringTok{"Turnover Field Position and Expected Points"}\NormalTok{,}
    \AttributeTok{x =} \StringTok{"Turnovers Over Course of the Game"}\NormalTok{,}
    \AttributeTok{y =} \StringTok{"Yards from Opponent\textquotesingle{}s Goal Line"}
\NormalTok{  )}
\end{Highlighting}
\end{Shaded}

\pandocbounded{\includegraphics[keepaspectratio]{Rams-Seahawks-Analysis-Week-11_files/figure-latex/turnover-plot-1.pdf}}
The visualization shows that field position varied significantly across
all of turnovers. While Seattle's single turnover and the Rams'
fourth-quarter interception both occurred around the 35-yard line in
opponent territory, the Rams' other three turnovers were deep in their
own half, between their 21 and 36-yard lines. This explains why, despite
forcing four turnovers, the Rams' expected points totaled only 5.6.

\begin{center}\rule{0.5\linewidth}{0.5pt}\end{center}

\section{Part 2: Win Probability
Analysis}\label{part-2-win-probability-analysis}

Now that we know the turnovers helped the Rams win the ball game, let's
examine when they happened and which ones had the biggest impact on win
probability. While all turnovers affect the game, not all turnovers are
created equal. Some occur in higher-leverage situations that drastically
swing the odds of winning. This section analyzes which specific plays
(turnovers or otherwise) had the greatest impact on the game's outcome.

\subsection{Win Probability Data}\label{win-probability-data}

\begin{Shaded}
\begin{Highlighting}[]
\CommentTok{\# Get plays with win probability data}
\NormalTok{wp\_data }\OtherTok{\textless{}{-}}\NormalTok{ game\_data }\SpecialCharTok{\%\textgreater{}\%} 
  \FunctionTok{filter}\NormalTok{(}\SpecialCharTok{!}\FunctionTok{is.na}\NormalTok{(posteam), posteam }\SpecialCharTok{!=} \StringTok{""}\NormalTok{) }\SpecialCharTok{\%\textgreater{}\%}
  \FunctionTok{filter}\NormalTok{(}\SpecialCharTok{!}\FunctionTok{is.na}\NormalTok{(wp)) }\SpecialCharTok{\%\textgreater{}\%}
  \FunctionTok{select}\NormalTok{(}
\NormalTok{    qtr, time, desc, posteam,}
\NormalTok{    home\_wp, away\_wp,}
\NormalTok{    wpa, ep, epa,}
\NormalTok{    interception, fumble\_lost, touchdown, field\_goal\_attempt, field\_goal\_result}
\NormalTok{  ) }\SpecialCharTok{\%\textgreater{}\%}
  \FunctionTok{mutate}\NormalTok{(}\AttributeTok{play\_number =} \FunctionTok{row\_number}\NormalTok{())}

\NormalTok{wp\_data}
\end{Highlighting}
\end{Shaded}

\begin{verbatim}
## # A tibble: 165 x 15
##      qtr time  desc  posteam home_wp away_wp      wpa    ep     epa interception
##    <dbl> <chr> <chr> <chr>     <dbl>   <dbl>    <dbl> <dbl>   <dbl>        <dbl>
##  1     1 15:00 5-J.~ LA        0.546   0.454  0.00623 1.84   0.300             0
##  2     1 14:55 (14:~ LA        0.552   0.448 -0.0182  2.14  -0.593             0
##  3     1 14:51 (14:~ LA        0.534   0.466  0.0502  1.55   1.91              0
##  4     1 14:19 (14:~ LA        0.584   0.416  0.0753  3.45   1.82              0
##  5     1 13:33 (13:~ LA        0.660   0.340 -0.00285 5.27  -0.159             0
##  6     1 12:52 (12:~ LA        0.657   0.343 -0.0239  5.11  -0.826             0
##  7     1 12:48 (12:~ LA        0.633   0.367 -0.00371 4.28  -0.0901            0
##  8     1 12:12 (12:~ LA        0.629   0.371 -0.101   4.19  -4.38              0
##  9     1 12:07 (12:~ SEA       0.529   0.471  0.0126  0.187  0.707             0
## 10     1 11:27 (11:~ SEA       0.516   0.484  0.00515 0.894  0.209             0
## # i 155 more rows
## # i 5 more variables: fumble_lost <dbl>, touchdown <dbl>,
## #   field_goal_attempt <dbl>, field_goal_result <chr>, play_number <int>
\end{verbatim}

\subsection{Biggest Win Probability
Swings}\label{biggest-win-probability-swings}

\begin{Shaded}
\begin{Highlighting}[]
\CommentTok{\# Taking a look at the ten plays that had the largest swings on win probability }
\NormalTok{biggest\_swings }\OtherTok{\textless{}{-}}\NormalTok{ wp\_data }\SpecialCharTok{\%\textgreater{}\%}
  \FunctionTok{filter}\NormalTok{(}\SpecialCharTok{!}\FunctionTok{is.na}\NormalTok{(wpa)) }\SpecialCharTok{\%\textgreater{}\%}
  \FunctionTok{arrange}\NormalTok{(}\FunctionTok{desc}\NormalTok{(}\FunctionTok{abs}\NormalTok{(wpa))) }\SpecialCharTok{\%\textgreater{}\%}
  \FunctionTok{head}\NormalTok{(}\DecValTok{10}\NormalTok{)}

\FunctionTok{print}\NormalTok{(biggest\_swings }\SpecialCharTok{\%\textgreater{}\%} \FunctionTok{select}\NormalTok{(qtr, time, posteam, wpa, ep, epa, desc))}
\end{Highlighting}
\end{Shaded}

\begin{verbatim}
## # A tibble: 10 x 7
##      qtr time  posteam     wpa      ep    epa desc                              
##    <dbl> <chr> <chr>     <dbl>   <dbl>  <dbl> <chr>                             
##  1     4 00:01 SEA     -0.388   1.32   -1.32  (:01) 5-J.Myers 61 yard field goa~
##  2     1 10:59 SEA     -0.182   1.10   -7.43  (10:59) (No Huddle) 14-S.Darnold ~
##  3     3 01:15 SEA     -0.173   1.25   -5.80  (1:15) 14-S.Darnold pass deep mid~
##  4     1 01:41 LA       0.134   2.38    4.18  (1:41) 23-K.Williams left tackle ~
##  5     4 00:05 SEA      0.116   0.607   0.711 (:05) (Shotgun) 14-S.Darnold pass~
##  6     4 10:55 SEA     -0.112   2.96   -5.84  (10:55) (Shotgun) 14-S.Darnold pa~
##  7     4 01:50 LA      -0.111  -0.0277  0.173 (1:50) 42-E.Evans punts 50 yards ~
##  8     1 12:12 LA      -0.101   4.19   -4.38  (12:12) 9-M.Stafford pass incompl~
##  9     3 01:28 LA      -0.0991 -0.110  -2.05  (1:28) 42-E.Evans punts 31 yards ~
## 10     3 14:09 LA      -0.0919  3.61   -5.21  (14:09) (Shotgun) 9-M.Stafford pa~
\end{verbatim}

\begin{Shaded}
\begin{Highlighting}[]
\NormalTok{biggest\_swings }\SpecialCharTok{\%\textgreater{}\%}
  \FunctionTok{mutate}\NormalTok{(}\AttributeTok{wpa\_pct =} \FunctionTok{paste0}\NormalTok{(}\FunctionTok{round}\NormalTok{(wpa }\SpecialCharTok{*} \DecValTok{100}\NormalTok{, }\DecValTok{1}\NormalTok{), }\StringTok{"\%"}\NormalTok{))}
\end{Highlighting}
\end{Shaded}

\begin{verbatim}
## # A tibble: 10 x 16
##      qtr time  desc  posteam home_wp away_wp     wpa      ep    epa interception
##    <dbl> <chr> <chr> <chr>     <dbl>   <dbl>   <dbl>   <dbl>  <dbl>        <dbl>
##  1     4 00:01 (:01~ SEA       0.612   0.388 -0.388   1.32   -1.32             0
##  2     1 10:59 (10:~ SEA       0.511   0.489 -0.182   1.10   -7.43             1
##  3     3 01:15 (1:1~ SEA       0.606   0.394 -0.173   1.25   -5.80             1
##  4     1 01:41 (1:4~ LA        0.649   0.351  0.134   2.38    4.18             0
##  5     4 00:05 (:05~ SEA       0.728   0.272  0.116   0.607   0.711            0
##  6     4 10:55 (10:~ SEA       0.814   0.186 -0.112   2.96   -5.84             1
##  7     4 01:50 (1:5~ LA        0.852   0.148 -0.111  -0.0277  0.173            0
##  8     1 12:12 (12:~ LA        0.629   0.371 -0.101   4.19   -4.38             0
##  9     3 01:28 (1:2~ LA        0.662   0.338 -0.0991 -0.110  -2.05             0
## 10     3 14:09 (14:~ LA        0.774   0.226 -0.0919  3.61   -5.21             0
## # i 6 more variables: fumble_lost <dbl>, touchdown <dbl>,
## #   field_goal_attempt <dbl>, field_goal_result <chr>, play_number <int>,
## #   wpa_pct <chr>
\end{verbatim}

While many plays impacted both teams' ability to win this game, the one
with the largest probability swing was the very last play, Seattle's
missed 61-yard field goal attempt, which swung win probability by
38.8\%. The fact that a team which turned the ball over four times still
had a chance to win on the final play speaks to Seattle's resilience on
both sides of the ball and LA's inability to put the game away. It's
worth noting that Myers had already made a 57-yard field goal earlier in
the game, and with his career long being 61 yards, this was a legitimate
scoring opportunity within his range. Beyond the missed kick, the
first-quarter interception and one third-quarter interception also
ranked among the top three plays in win probability impact, with swings
of 18.2\% and 17.3\% respectively.

\subsection{Visualization: Win Probability
Flow}\label{visualization-win-probability-flow}

\begin{Shaded}
\begin{Highlighting}[]
\FunctionTok{ggplot}\NormalTok{(wp\_data, }\FunctionTok{aes}\NormalTok{(}\AttributeTok{x =}\NormalTok{ play\_number)) }\SpecialCharTok{+}
  \FunctionTok{geom\_line}\NormalTok{(}\FunctionTok{aes}\NormalTok{(}\AttributeTok{y =}\NormalTok{ home\_wp, }\AttributeTok{color =} \StringTok{"LA Rams"}\NormalTok{), }\AttributeTok{size =} \FloatTok{1.5}\NormalTok{) }\SpecialCharTok{+}
  \FunctionTok{geom\_line}\NormalTok{(}\FunctionTok{aes}\NormalTok{(}\AttributeTok{y =}\NormalTok{ away\_wp, }\AttributeTok{color =} \StringTok{"SEA Seahawks"}\NormalTok{), }\AttributeTok{size =} \FloatTok{1.5}\NormalTok{) }\SpecialCharTok{+}
  \FunctionTok{geom\_hline}\NormalTok{(}\AttributeTok{yintercept =} \FloatTok{0.5}\NormalTok{, }\AttributeTok{linetype =} \StringTok{"dashed"}\NormalTok{, }\AttributeTok{alpha =} \FloatTok{0.5}\NormalTok{) }\SpecialCharTok{+}
  \FunctionTok{scale\_color\_manual}\NormalTok{(}\AttributeTok{values =} \FunctionTok{c}\NormalTok{(}\StringTok{"LA Rams"} \OtherTok{=} \StringTok{"navy"}\NormalTok{, }\StringTok{"SEA Seahawks"} \OtherTok{=} \StringTok{"green"}\NormalTok{)) }\SpecialCharTok{+}
  \FunctionTok{scale\_y\_continuous}\NormalTok{(}\AttributeTok{labels =}\NormalTok{ scales}\SpecialCharTok{::}\FunctionTok{percent\_format}\NormalTok{()) }\SpecialCharTok{+}
  \FunctionTok{labs}\NormalTok{(}
    \AttributeTok{title =} \StringTok{"Win Probability: LA Rams vs Seattle Seahawks"}\NormalTok{,}
    \AttributeTok{x =} \StringTok{"Play Number"}\NormalTok{,}
    \AttributeTok{y =} \StringTok{"Win Probability"}
\NormalTok{  )}
\end{Highlighting}
\end{Shaded}

\pandocbounded{\includegraphics[keepaspectratio]{Rams-Seahawks-Analysis-Week-11_files/figure-latex/win-prob-chart-1.pdf}}

Looking at the chart, we can see that LA had a win probability of over
50\% for the entire game and it peaked around 85\% somewhere in the
third or fourth quarter. The game started out being very competitive,
hovering close to 50/50. Then, following what we know about how this
game flowed, it seems the Rams took control likely after that
first-quarter interception. It is worth noting that LA's win probability
drops from around 75\% to 65\% on the final play, while Seattle's line
shoots up from around 25\% to 38\%. Despite being dominated most of the
game, the Seahawks had a realistic shot at the end, as evidenced by the
final spike in the green line showing Seattle's win probability reaching
38\% on the missed field goal attempt.

\begin{center}\rule{0.5\linewidth}{0.5pt}\end{center}

\section{Key Findings}\label{key-findings}

This analysis tried to get to the bottom of two questions: Did the Rams
capitalize on their takeaways, and how did the turnovers affect the flow
of the game and win probability? The data showed us that 2 out 4 drives
following a Rams takeaway, they were able to go down and score a
touchdown resulting in 14 points. Which is surprising given that they
were only expected 5.6 points based on field position. Additionally, the
game started off being very competitive, but after that first Sam
Darnold interception, the win probability swung in the Rams' favour and
they were projected to win for the rest of the ball game.

This game was close for several reasons. Mainly, 3 out of 4 times after
the Rams got a takeaway, they started in their own territory and had to
drive close to the length of the field, limiting their expected point
totals despite generating turnovers. Additionally, the Rams' offense
struggled outside of turnover opportunities, scoring only 7 points on
all other drives, revealing their heavy dependence on defensive
playmaking. Meanwhile, the Seahawks converted their single takeaway into
a field goal and remained within striking distance throughout. Despite
trailing for most of the game, Seattle had a legitimate chance to win on
the final play with a 61-yard field goal attempt that matched kicker
Jason Myers' career long.

\begin{center}\rule{0.5\linewidth}{0.5pt}\end{center}

\section{Conclusion}\label{conclusion}

This analysis shows that while turnovers are a major driver of game
outcomes, their impact is highly context dependent. Despite turning the
ball over four times, Seattle remained competitive and had a realistic
chance to win on the final play. This underscores how factors such as
field position, offensive efficiency, and situational leverage can
offset even significant turnover disadvantages.

Although the Rams forced four takeaways, they struggled to consistently
convert those opportunities into points. Outside of drives following
turnovers, their offense produced just seven points, highlighting a
reliance on defensive playmaking rather than sustained offensive
efficiency.

Future analysis could focus on red zone performance, third-down
efficiency, and defensive scheme usage to better understand why both
offenses stalled frequently and how specific tactical decisions
influenced scoring outcomes.

\begin{center}\rule{0.5\linewidth}{0.5pt}\end{center}

\end{document}
